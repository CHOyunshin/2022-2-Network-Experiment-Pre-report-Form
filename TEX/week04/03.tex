\section{데이터셋을 Train, Test, Valid 로 나누는 이유}
\subsection{Train set과 Test set을 나누는 이유}
    머신러닝 모델을 적용하면서 학습 알고리즘이 최적의 효과를 발휘하기 위해서는 적절한 데이터 전처리를 해야한다. 학습의 parameter가 되는 featrue를 선택해주는 작업부터, featrue등의 scaling 까지의 데이터 전처리를 거치면서 처리된 데이터들이 학습알고리즘에서 잘 작동하는지 확인하는지 확인하는 과정이 필요하다.\\
    \vspace{-4mm}
    \begin{figure}[!h]\centering
		\includegraphics[width=.65\textwidth]{image/week04/3-1.png}
		\caption{\small A roadmap for building machine learning systems}
		\vspace{-10pt}
    \end{figure}
    
    학습하는 머신러닝 알고리즘이 Train dataset에서 잘 작동하고 새로운 dataset에 대해서도 잘 일반화되는지를 확인하기 위해서 데이터셋을 랜덤하게 Train dataset 과 Test dataset 으로 나누어준다. 

    Figure에서 확인할 수 있듯이 Train dataset 을 이용해서 머신러닝 모델을 훈련, 최적화하고 Test dataset은 별도로 보관후 알고리즘의 Learning이 끝난후 최종 모델의 Evaluation 단계에서 최종 모델 일반화의 평가에 사용한다.
\subsection{Train data를 Train dataset 과 Valid dataset으로 나누는 이유}
    지도학습을 통해서 분류에 사용되는 알고리즘 모델들은 각각의 태생적 편향성을 가지고 있고, 이때문에 최적의 성능을 가진 모델은 절대치 보다는 학습하고자 하는 dataset의 특성에 종류에 따라서 성능이 나뉜다고 볼 수 있다. 따라서 최적의 학습모델을 적용시키기 위해서는 여러 모델들을 선택해서 비교해야 하는데, 이때 모델의 성능을 평가하는 지표는 정확도이다. 

    이 과정에서 Final Model 을 평가하기 위해서 별도로 분류한 Test dataset 이외에도 Learning 과정에서 모델의 성능을 평가할 다양한 교차검증기법을 적용하는데 새로운 비교군이 필요하다. 따러서 교차검증에서의 모델의 일반화 성능을 예측하기 위해서 Train data를 Train dataset 과 Valid dataset 으로 나누어 준다.
\subsection{Train 과 Test의 적절한 비율에 관하여}
    Dataset을 Train 과 Test로 나누는 부분에서 어느정도 데이터의 열화가 발생하게 된다. 하지만 Test set을 너무 작게 뽑을 경우 Final Model에 대한 일반화 오차에 대한 추정이 부정확해 질 것이다. 
    
    실전에서 사용하는 경우 일반적인 dataset에 대해서는 70:30, 80:20이 많이 사용되고 10만개 이상의 훈련  sample이 있는 dataset의 경우 90:10, 99:1 의 분할을 주로 사용한다고 한다.
    
    뿐만아니라 떼어 놓았던 test dataset을 evaluation 이후에 다시 training set으로 재활용하는 방법 또한 일반적으로 사용된다. 일반적인 경우에서는 권장되지만 dataset의 크기가 작고 dataset에 이상치값을 많이 포함하는 경우라면 오히려 일반화성능이 나빠지는것 또한 고려해야된다.
\clearpage