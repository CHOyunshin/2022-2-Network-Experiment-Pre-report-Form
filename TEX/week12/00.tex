\subsection*{INTRO}
이번주차 실험에서는 MAC (Medium Access Control) 을 다룬다.  Data Link Layer에서 다루는 기능인 data link control에서 error detectiion 과 재전송등의 process를 관장하는데, 네트워크에서 한 회선에서 여러 디바이스가 공유하는 상황에서 어떻게 control 하는지 알아본다. 이러한 회선을 공유하는 상황에서 2개이상의 신호가 겹치면 데이터의 수신율이 떨어진다. 즉 shared link에서 한 디바이스만이 전달을 하게 하는것이 multiple access resolution이고 이때 사용하는 protocol을 MAC 이라고 한다. 그중접속하는 각 디바이스의 전달시간을 random 시간에 의존하여 collison을 회피하는 protocol을 Random Access Protocols 라고한다. 이중 ALOHA, CSMA, CSMA/CA에 대해서 다루도록 한다.
