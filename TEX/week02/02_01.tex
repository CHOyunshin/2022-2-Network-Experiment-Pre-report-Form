\section{IQ Signal}
\subsection{Quadrature siganl}
    \begin{equation*}
        e^{jx} = I + Q
    \end{equation*}
    Quadrature signal은  복소수 체계를 기반으로 하기에 complex signal(복소 신호)라고도 불리며, I와 Q 2개로 이루어져 있으므로 I/Q 신호라고도 자주 불린다. Quadrature signal이 사용되는 신호처리를 quadrature processing이라고 한다.
    Quadrature signal은 특정 시간에서의 값을 하나의 복소수로 표현할 수 있는 2차원 신호"라고 수학적 으로 정의된다. 이때 각각을 실수부와 허수부를 in-phase와 quadrature phase라고 부르며 이 단어의 앞 글자를 따서 각각 $I$, $Q$라 한다.
    
\subsection{Quadrature Signal의 Time-Domain 에서 표현}
    Euler Equation으로 complex number를 표현할시에, magnitude $M = 1$이라하고, $\Phi = \omega t, \omega = 2 \pi f$로 하면 다음의 식을 얻을 수 있다. 
    \begin{equation*}
        c = e^{j2\pi f_0 t}
    \end{equation*}
    아래의 그림과 같이 복소수 c를 (a)의 점 또는 (b)의 벡터의 형태로 나타낼 수 있다. c는 (a)의 파란색 점과 같이 나타나는데, 시간 $t$가 증가하면 반시계 방향으로 이동한다. 주파수 $f$가 1일때 점은 1초에 1바퀴를 돌게된다. $f$ 대신 $-f$를 대입할 경우 음의 부호가 있는 하얀색 점과 같이 나타나는데, 파란색 점과 반대로 $t$가 증가하면 시계방향으로 이동한다.
    
    \vspace{-4mm}  
    \begin{figure}[!h]\centering
		\includegraphics[width=.75\textwidth]{image/week02/2-1-1.png}
		\caption{\small Quadrature Signal Plot}
		\vspace{-10pt}
    \end{figure}
    
    위의 두가지 quadrature signal을 이용해 sine 함수와 cosine 함수를 유도할 수 있다. cosine 함수는 두 signal의 합, sine 함수는 두 signal의 차이다. 이를 응용하여 모든 quadrature signal은 sine, cosine 함수의 합으로 나타낼 수 있고 반대도 성립한다.
    \begin{equation*}
        2\cos(2\pi f_0t) = e^{j2\pi f_0 t} + e^{-j2\pi f_0 t}\\
        2\sin(2\pi f_0t) = e^{j2\pi f_0 t} - e^{-j2\pi f_0 t}
    \end{equation*}
    Quadrature signal은 2차원 신호이다. 여기에 시간이라는 축을 추가하면 다음과 같이 3차원 그래프로 나타낼 수 있다.\\
    
    \vspace{-4mm}  
    \begin{figure}[!h]\centering
		\includegraphics[width=.75\textwidth]{image/week02/2-1-2.png}
		\caption{\small Quadrature Signal in Time Domain}
		\vspace{-10pt}
    \end{figure}
    
    왼쪽의 figure는 시간에 따라 변하는 phasor를 vector로 표현한 것이고, 오른쪽 figure는 phasor의 시간에 따른 연속적인 변화를 표현한 것이다. 우측 figure에서 아래와 우측 평면에 cosine 함수와 sine 함수를 찾을 수 있는데 이 두 함수가 합쳐져 3차원 나선 모양의 quadrature signal을 만든 것이다.
    
\newpage
\subsection{Quadrature Signal의 Frequency-Domain 에서 표현}
    이번에는 주파수 영역에서 quadrature signal의 성질을 확인해보자. 아래 그림은 복소평면에 각각 시간 축과 주파수 축을 추가한 것이다.
    
    \vspace{-4mm}  
    \begin{figure}[!h]\centering
		\includegraphics[width=.75\textwidth]{image/week02/2-2-1.png}
		\caption{\small Quadrature Signal in Time Domain and Frequency Domain}
		\vspace{-10pt}
    \end{figure}
    
    시간 축으로 나타낸 좌측 그림에는 cosine 함수와 sine 함수가 허수 부분이 없이 실수로 나타나 있다. 우측 그림은 같은 cosine 함수와 sine 함수를 주파수 축에 나타낸 그림인데, 두개의 주파수 $f_0$ 과 $-f_0$에 impulse 형태로 표현된다. cosine 함수는 실수 축의 양의 방향으로 2개의 impulse가, sine 함수는 허수 축으로 양의 방향과 음의 방향 각각 1개의 impulse가 나타난다.

    이 두 함수를 응용하여 여러가지 quadrature signal을 만들 수 있다. 다음의 그림과 같이 sine 함수에 j를 곱하여 반시계 방향으로 회전하여 실수 방향의 impulse를 만들 수 있다. 여기에 cosine 함수를 더하면 음의 방향의 impulse가 상쇄되어 하나의 impulse만 남게된다. 이는 Euler Equation에서 한가지 주파수 $f_0$을 갖는 것을 주파수 영역에 나타낸것과 동일하다.
    
    \vspace{-4mm}  
    \begin{figure}[!h]\centering
		\includegraphics[width=.75\textwidth]{image/week02/2-2-2.png}
		\caption{\small Addition of two Quadrature Signals}
		\vspace{-10pt}
    \end{figure}
    
    다음의 그림의 (a)는 cosine함수에 위상 $\phi$를 추가한 것을 주파수 영역에 나타낸 것이다. 원래 cosine 함수는 실수 성분만을 갖지만 위상 $\phi$가 추가됐을때 허수 성분이 추가되어 방향이 기울어지게 된다.
    
    \vspace{-4mm}  
    \begin{figure}[!h]\centering
		\includegraphics[width=.75\textwidth]{image/week02/2-2-3.png}
		\caption{\small Quadrature Signal with Phase}
		\vspace{-10pt}
    \end{figure}
    
    그림 (b), (c), (d)는 여러개의 phasor가 연속적으로 연결되어 대역대(bandwidth)를 이루는 신호들의 그림이다.

    주파수 대역을 바꾸는 것을 quadrature mixing 또는 complex mixing이라고 하는데 이를 주파수 영역에 나타내면 다음과 같다.
    
    \vspace{-4mm}  
    \begin{figure}[!h]\centering
		\includegraphics[width=.75\textwidth]{image/week02/2-2-4.png}
		\caption{\small Quadrature Mixing}
		\vspace{-10pt}
    \end{figure}
    
    (a)와 같이 주파수 $f_0$에 있는 신호에 $e^{j2\pi f_0 t}$를 곱하면 (b)와 같이 오른쪽으로 $f_0$만큼 이동한다. 반대로 $e^{-j2\pi f_0 t}$를 곱하면 (c)와 같이 왼쪽으로 $f_0$만큼 이동한다.