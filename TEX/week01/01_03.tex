\subsection{window의 의미}
TCP의 기능중 전송되는 데이터의 완전성 보장이외에도 혼잡제어 기능을 수행한다.  네트워크의 특성상 어느 경로에서 지연이 발생하는지 파악하기 어렵고,  지연이 반복되면 재전송등 비효울이 발생하고 지연이 중첩될 수 있다. 따라서 네트워크의 혼잡상태를 감지해서, 자연을 회피하기 위해 송신측의 윈도우 크기를 조절함으로서 데이터 전송량을 강제로 줄이는 혼잡제어 기능을 수행한다.

이떄 TCP가 인지하는 parametersms 송신 측은 자신의 최종 윈도우 크기를 정할 때 수신 측이 보내준 윈도우 크기인 수신자 \textbf{윈도우(RWND)}, 그리고 자신이 네트워크의 상황을 고려해서 정한 윈도우 크기인 \textbf{혼잡윈도우(CWND)} 중에서 더 작은 값을 사용한다.

즉 이떄 protocol이 설정해주는 송신측이 가지는 혼잡윈도우\footnote{혼잡 윈도우(Congestion Window, CWND)}의 크기를 window라고 한다.